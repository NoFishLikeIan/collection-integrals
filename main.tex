\documentclass{article}

\usepackage{fancyhdr}
\usepackage{extramarks}
\usepackage{amsmath}
\usepackage{amsthm}
\usepackage{amssymb}
\usepackage{amsfonts}
\usepackage{tikz}
\usepackage[plain]{algorithm}
\usepackage{algpseudocode}
\usepackage{mathpazo}
\usepackage{systeme}
\usepackage{multirow,array}
\usepackage{xcolor}
\usepackage{tikz-qtree}
\usepackage[long]{optidef}
\usepackage{apacite}
\usepackage{mathtools}
\usepackage[english]{babel}
\usepackage[utf8x]{inputenc}
\usepackage{amssymb}
\usepackage{csquotes}
\usepackage{subcaption}



\newcommand{\R}{\mathbf{R}}  
\newcommand{\Z}{\mathbf{Z}}
\newcommand{\N}{\mathbf{N}}
\newcommand{\Q}{\mathbf{Q}}
\newcommand*{\QEDA}{\hfill\ensuremath{\blacksquare}}%

\tikzset{main node/.style={circle,draw,minimum size=1.5cm,inner sep=0pt},
            }


\usetikzlibrary{automata,positioning}

%
% Basic Document Settingshttps://www.overleaf.com/6212182343ggsvjzswgdnx
%

\topmargin=-0.45in
\evensidemargin=0in
\oddsidemargin=0in
\textwidth=6.5in
\textheight=9.0in
\headsep=0.25in

\linespread{1.1}

\pagestyle{fancy}
\rhead{\firstxmark}
\lfoot{\lastxmark}
\cfoot{\thepage}

\renewcommand\headrulewidth{0.4pt}
\renewcommand\footrulewidth{0.4pt}

\setlength\parindent{0pt}
\setlength{\parskip}{1em}


%
% Create Problem Sections
%


\DeclarePairedDelimiter\floor{\lfloor}{\rfloor}


\newenvironment{theorem}[2][Theorem]{\begin{trivlist}
\item[\hskip \labelsep {\bfseries #1}\hskip \labelsep {\bfseries #2.}]}{\end{trivlist}}
\newenvironment{lemma}[2][Lemma]{\begin{trivlist}
\item[\hskip \labelsep {\bfseries #1}\hskip \labelsep {\bfseries #2.}]}{\end{trivlist}}
\newenvironment{claim}[2][Claim]{\begin{trivlist}
\item[\hskip \labelsep {\bfseries #1}\hskip \labelsep {\bfseries #2.}]}{\end{trivlist}}
\newenvironment{problem}[2][Problem]{\begin{trivlist}
\item[\hskip \labelsep {\bfseries #1}\hskip \labelsep {\bfseries #2.}]}{\end{trivlist}}
\newenvironment{proposition}[2][Proposition]{\begin{trivlist}
\item[\hskip \labelsep {\bfseries #1}\hskip \labelsep {\bfseries #2.}]}{\end{trivlist}}
\newenvironment{corollary}[2][Corollary]{\begin{trivlist}
\item[\hskip \labelsep {\bfseries #1}\hskip \labelsep {\bfseries #2.}]}{\end{trivlist}}
\newcommand*\Eval[3]{\left.#1\right\rvert_{#2}^{#3}}
\newenvironment{solution}{\begin{proof}[Solution]}{\end{proof}}



%
% Homework Problem Environment
%
% This environment takes an optional argument. When given, it will adjust the
% problem counter. This is useful for when the problems given for your
% assignment aren't sequential.
%
\newenvironment{homeworkProblem}[1][-1]{
    \ifnum#1>0
        \setcounter{homeworkProblemCounter}{#1}
    \fi
    \section{Problem \arabic{homeworkProblemCounter}}
    \setcounter{partCounter}{1}
}{
}

\newcommand{\hmwkTitle}{A collection of integrals}
\newcommand{\hmwkAuthorName}{\textbf{NoFishLikeIan}}

%
% Title Page
%

\title{
    \vspace{2in}
    \textmd{\hmwkTitle}\\
    \vspace{3in}
}

\author{\hmwkAuthorName}

\renewcommand{\part}[1]{\textbf{\large Part \Alph{partCounter}}\stepcounter{partCounter}\\}

%
% Various Helper Commands
%

% Useful for algorithms
\newcommand{\alg}[1]{\textsc{\bfseries \footnotesize #1}}

% For derivatives
\newcommand{\deriv}[1]{\frac{\mathrm{d}}{\mathrm{d}x} (#1)}

% For partial derivatives
\newcommand{\pderiv}[2]{\frac{\partial}{\partial #1} (#2)}

% Integral dx
\newcommand{\dx}{\mathrm{d}x}


\newcommand{\norm}[1]{\left\lVert#1\right\rVert}
\newcommand{\abs}[1]{\left\lvert#1\right\rvert}
%\newcommand{\norm}[1]{\lVert#1\rVert_2}

% Alias for the Solution section header
%\newcommand{\solution}{\textbf{\large Solution}}

% Probability commands: Expectation, Variance, Covariance, Bias
\newcommand{\E}{\mathrm{E}}
\newcommand{\Var}{\mathrm{Var}}
\newcommand{\Cov}{\mathrm{Cov}}
\newcommand{\Bias}{\mathrm{Bias}}
\usepackage{accents}
\newcommand{\ubar}[1]{\underaccent{\bar}{#1}}
\newcommand*\diff{\mathop{}\!\mathrm{d}}

\newcommand{\citeout}[1]{\citeauthor{#1} \citeyear{#1}}

\begin{document}

\maketitle
\thispagestyle{empty}

\newpage

\setcounter{page}{1}

\section{Rabee's formula}

\begin{equation} \label{It}
    I(t) \coloneqq \int^{t+1}_{t} \ln\left(\Gamma(z)\right) \diff z
\end{equation}

Let $x = z- t \implies \diff z = \diff x$, then

\begin{equation}
    \begin{split}
        I(t) &= \int^1_0 \ln\left(\Gamma(x + t)\right) \diff x \\
        \text{By Leibniz rule}\\
        I'(t) &= \int^1_0 \partial_t\left[\ln\left(\Gamma(x + t)\right)\right] \diff x \\
        &= \int^1_0 \frac{1}{\Gamma(x + t)} \Gamma'(x + t) \diff x 
    \end{split}
\end{equation}

Note that the partial derivative with respect to $t$ is equivalent to that with respect to $x$,

\begin{equation}
    \begin{split}
        I'(t) &\equiv \frac{\partial I(t)}{\partial x} \\
        &= \ln(\Gamma(x + t)) \vert^{x=1}_{x=0} \\
        &= \ln(\Gamma(1 + t)) - \ln(\Gamma(t))\\
        &= \ln\left(\frac{\Gamma(t+1)}{\Gamma(t)}\right) \\
        &= \ln\left(\frac{\Gamma(t) \cdot t}{\Gamma(t)}\right) = \ln(t)
    \end{split}
\end{equation}

By integrating with respect to $t$

\begin{equation} \label{tau_int}
\begin{split}
   \int^{\tau}_{0} I'(t) \diff t &= I(\tau) -  I(0) \\
    &= \int^{\tau}_{0} \ln(t) \diff t \\
    &= \ln(t) \cdot t - t \vert^{\tau}_{0} \\
    &= \ln(\tau) \cdot \tau - \tau - \lim_{t \xrightarrow{} 0} \ln(t) \cdot t \\
    &= \ln(\tau) \cdot \tau - \tau
\end{split}
\end{equation}

Hence we can rewrite (\ref{tau_int})

\begin{equation}
    I(\tau) = I(0) + \ln(\tau) \cdot \tau - \tau
\end{equation}

But we need to find $I(0)$. We can evaluate (\ref{It}) at 0, then

\begin{equation}
    J \coloneqq I(0) = \int^1_0 \ln(\Gamma(x)) \diff x \\
\end{equation}

Let now $x= 1-y \implies \diff x = -\diff y$ in order to preserve the bounds, then

\begin{equation}
    \begin{split}
        J &= \int^0_1 \ln(\Gamma(1 - y)) (-\diff y) \\
        &=\int^1_0\ln(\Gamma(1 - y)) \diff y
    \end{split}
\end{equation}

Note that this implies that

\begin{equation}
    \begin{split}
        J &= \frac{1}{2} \cdot \int^1_0 \ln(\Gamma(x)) \ln(\Gamma(1 - X)) \diff x \\
        &= \frac{1}{2} \cdot \int^1_0 \ln(\Gamma(x) \cdot \Gamma(1 - X)) \diff x\\
        &\text{by Euler's reflection formula}\\
        &= \frac{1}{2} \cdot \int^1_0 \ln\left( \frac{\pi}{\sin{(\pi \cdot x)}} \right) \diff x \\
        &= \frac{1}{2} \cdot \int^1_0 \ln(\pi) \diff x - \frac{1}{2} \cdot \int^1_0 \ln(\sin{(\pi \cdot x)}) \diff x 
    \end{split}
\end{equation}

Let $z = \pi \cdot x \implies \diff x = \frac{\diff z}{\pi}$,

\begin{equation}
    \begin{split}
        J &= \frac{1}{2} \cdot \int^1_0 \ln(\pi) \diff x - \frac{1}{2 \pi} \cdot \int^\pi_0 \ln(\sin{(z)}) \diff z \\
        &= \frac{1}{2} \cdot \int^1_0 \ln(\pi) \diff x + \frac{1}{2} \cdot \ln(2) = \frac{1}{2} \ln(2 \cdot \pi)
    \end{split}
\end{equation}

Then plugging back in

\begin{equation}
    \int^1_0 \ln\left(\Gamma(x + t)\right) \diff x = \frac{\ln(2\pi)}{2} + \ln(t) \cdot t -t 
\end{equation}

\section{Wallis product}

By using Euler product form of $sin(x)$ we can write

\begin{equation}
    sin(x) = x\cdot \prod^{\infty}_{n=1} \left( 1 - \frac{x^2}{n^2 \pi^2} \right) 
\end{equation}

Then evaluate 

\begin{equation}
\begin{split}
    sin(\pi / 2) &= \frac{\pi}{2} \cdot \prod^{\infty}_{n=1}\left( 1 - \frac{1}{(2n)^2} \right) \\
    \frac{2}{\pi} &= \prod^{\infty}_{n=1}\left( \frac{(2n)^2 - 1}{(2n)^2}  \right) \\
    \frac{\pi}{2} &= \prod^{\infty}_{n=1} \left( \frac{(2n)^2}{(2n)^2 - 1} \right) \\
    \frac{\pi}{2} &= \prod^{\infty}_{n=1} \left( \frac{2n}{2n - 1} \cdot \frac{2n}{2n + 1} \right) = \frac{2}{1} \cdot \frac{2}{3} \cdot \frac{4}{3} \cdot \frac{4}{5} \dots
\end{split}
\end{equation}

\end{document}